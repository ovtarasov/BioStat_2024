\documentclass[12pt, a4paper]{article}
\usepackage{amssymb,amsmath}
\usepackage{blindtext}
\usepackage{cmap}
%\usepackage[utf8x,utf8]{inputenc}
\usepackage[T2A]{fontenc}
\usepackage[english,russian]{babel}
% \usepackage{fixltx2e} % provides \textsubscript
% use upquote if available, for straight quotes in verbatim environments
\IfFileExists{upquote.sty}{\usepackage{upquote}}{}
% use microtype if available
\IfFileExists{microtype.sty}{%
\usepackage{microtype}
\UseMicrotypeSet[protrusion]{basicmath} % disable protrusion for tt fonts
}{}
\usepackage[margin=1in]{geometry}
\usepackage{hyperref}
\hypersetup{colorlinks=true,
            urlcolor=blue,
            unicode=true,
            pdftitle={Домашнее задание по теории вероятностей №2},
            pdfauthor={Олег Тарасов},
            pdfborder={0 0 0},
            breaklinks=true}
\urlstyle{same}  % don't use monospace font for urls
\usepackage{graphicx,grffile}
\makeatletter
\def\maxwidth{\ifdim\Gin@nat@width>\linewidth\linewidth\else\Gin@nat@width\fi}
\def\maxheight{\ifdim\Gin@nat@height>\textheight\textheight\else\Gin@nat@height\fi}
\makeatother
% Scale images if necessary, so that they will not overflow the page
% margins by default, and it is still possible to overwrite the defaults
% using explicit options in \includegraphics[width, height, ...]{}
\setkeys{Gin}{width=\maxwidth,height=\maxheight,keepaspectratio}
\IfFileExists{parskip.sty}{%
\usepackage{parskip}
}{% else
\setlength{\parindent}{0pt}
\setlength{\parskip}{6pt plus 2pt minus 1pt}
}
\setlength{\emergencystretch}{3em}  % prevent overfull lines
\providecommand{\tightlist}{%
  \setlength{\itemsep}{0pt}\setlength{\parskip}{0pt}}
\setcounter{secnumdepth}{0}
% Redefines (sub)paragraphs to behave more like sections
\ifx\paragraph\undefined\else
\let\oldparagraph\paragraph
\renewcommand{\paragraph}[1]{\oldparagraph{#1}\mbox{}}
\fi
\ifx\subparagraph\undefined\else
\let\oldsubparagraph\subparagraph
\renewcommand{\subparagraph}[1]{\oldsubparagraph{#1}\mbox{}}
\fi

%%% Use protect on footnotes to avoid problems with footnotes in titles
\let\rmarkdownfootnote\footnote%
\def\footnote{\protect\rmarkdownfootnote}

%%% Change title format to be more compact
\usepackage{titling}

% Create subtitle command for use in maketitle
\newcommand{\subtitle}[1]{
  \posttitle{
    \begin{center}\large#1\end{center}
    }
}

\setlength{\droptitle}{-2em}

  \title{Домашнее задание по теории вероятностей №2}
    \pretitle{\vspace{\droptitle}\centering\large\textbf}
  \posttitle{\par}
    \author{Олег Тарасов}
    \preauthor{\centering\textit}
  \postauthor{\par}
      \predate{\centering\textit}
  \postdate{\par}
    \date{2024-10-13}

\parskip=4pt               % интервал между абзацами
\textheight=25cm           % высота текста
\textwidth=17cm            % ширина текста
\oddsidemargin=0pt         % отступ от левого края
%\topmargin=-1.5cm          % отступ от верхнего края

\begin{document}
\maketitle

\paragraph{Задание 1.}\label{-1.}

\textit{\textbf{Доказать, что из \(Pr\{A\}=Pr\{A | B\}\) автоматически
следует, что \(Pr\{A\} = Pr\{A | \overline{B}\}\).}}

Преобразуем условную вероятность \[
Pr\{A | \overline{B} \} = \frac{Pr\{A\overline{B}\} }{Pr\{\overline{B}\} } = \frac{Pr\{A\} - Pr\{AB\} }{Pr\{\overline{B}\} } = ...
\]

По условию \(Pr\{A\} = Pr\{A | B\}\), поэтому выражение можно
преобразовать далее как \[
...= \frac{Pr\{A | B\} - Pr\{AB\} }{Pr\{\overline{B}\} } = \frac{ Pr\{AB\}/Pr\{B \} - Pr\{AB\} }{Pr\{\overline{B}\} }
\]

Вынеся \(Pr\{ B\}\) в общий знаменатель, получаем \[
\frac{ Pr\{AB\} - Pr\{AB\}Pr\{B \} }{Pr\{B \}Pr\{\overline{B}\} } = \frac{ Pr\{AB\}\big(1-Pr\{B \}\big) }{Pr\{B \}Pr\{\overline{B}\} } = \frac{ Pr\{AB\}Pr\{\overline{B} \} }{Pr\{B \}Pr\{\overline{B}\} } = \frac{ Pr\{AB\} }{Pr\{B \} } = \] 
\(= Pr\{A | B\}\), что по условию равняется \(Pr\{A\}\).

\begin{center}
    \textit{\textbf{QED}}    
\end{center}

\paragraph{Задание 2.}\label{-2.}

\textit{\textbf{Доказать, что из \(RR = 1\) следует, что случайные события
-- независимы.}}

Отношение рисков наблюдать событие \(A\) при условии наступления
либо не наступления события \(B\) равняется \(RR = \frac{Pr\{A | B\} }{Pr\{A | \overline{B} \} }=1\), следовательно, 
\(Pr\{A | B\} = Pr\{A | \overline{B} \}\).

Представим условные вероятности в виде \[
Pr\{A | B\} = \frac{Pr\{AB\} }{Pr\{B\} }
\]

\[
Pr\{A | \overline{B} \} = \frac{Pr\{A\overline{B}\} }{Pr\{\overline{B}\} } = \frac{Pr\{A\} - Pr\{AB\} }{Pr\{\overline{B}\} } = 
\frac{Pr\{A\} }{Pr\{\overline{B}\} } - \frac{Pr\{AB\}}{Pr\{\overline{B}\} }
\]

Приравняв эти два выражения и перенеся вычитаемое через знак равенства, получаем \[
\frac{Pr\{A\} }{Pr\{\overline{B}\} } = \frac{Pr\{AB\} }{Pr\{B\} } + \frac{Pr\{AB\}}{Pr\{\overline{B}\} } 
\]

Преобразуя правую часть уравнения, получаем \[
Pr\{AB\} \left( \frac{1}{Pr\{B\} } + \frac{1}{Pr\{\overline{B}\} } \right) = Pr\{AB\} \frac{Pr\{B\} + Pr\{\overline{B}\} } {Pr\{B\} Pr\{\overline{B}\} } 
= \frac{Pr\{AB\} } {Pr\{B\} Pr\{\overline{B}\} }
\]

Поскольку \(Pr\{B\} \neq 0\) и \(Pr\{\overline{B}\} \neq 0\), то, приравняв
два предыдущих выражения и домножив на \(Pr\{B\}Pr\{\overline{B}\}\),
получаем \[
Pr\{A\}Pr\{B\} = Pr\{AB\},
\] что соответствует определению независимых событий.

\begin{center}
    \textit{\textbf{QED}}    
\end{center}

\paragraph{Задание 3.}\label{-3.}

\textit{\textbf{A. Найдите математическое ожидание и дисперсию числа
циклов терапии при первичном выявлении и при рецидиве (отдельно).}}

При первичном выявлении заболевания математическое ожидание количества
циклов химиотерапии \(\mu_1 = 1*0,5 + 2*0,5 = 1,5\), а дисперсия
\(\sigma^2_1 = \big( (1-1,5)^2*0,5 + (2-1,5)^2*0,5\big) = 0,25\).

При рецидиве математическое ожидание количества циклов терапии 
\(\mu_2 = 2*0,25 + 3*0,75 = 2,75\), а дисперсия
\(\sigma^2_2 = \big( (2-2,75)^2*0,25 + (3-2,75)^2*0,75\big) = 0,1875\).

\textit{\textbf{B. Предположим, что мы изучаем только рецидивировавших
пациентов.}}

\begin{itemize}
\tightlist
\item
  \textit{\textbf{Постройте таблицу распределения общего числа циклов
  терапии у рецидивировавших пациентов («дебютных» + «рецидивных»).}}
\end{itemize}

Если допустить, что выбор числа циклов при рецидиве не зависит от того,
сколько циклов было в дебюте, то таблица имеет следующий вид:

\begin{table}[h!]
    \centering
    \begin{tabular}{l|ccc}
        \hline
        Количество циклов & 3 & 4 & 5\\
        \hline
        Вероятность & 0,125 & 0,5 & 0,375\\
        \hline
    \end{tabular}
    % \caption{Caption}
    \label{tab:my_label}
\end{table}

\begin{itemize}
\tightlist
\item
  \textit{\textbf{Найдите математическое ожидание и дисперсию этой
  величины.}}
\end{itemize}

Поскольку новая случайная величина является суммой двух исходных, и мы принимаем допущение об их независимости, то математическое ожидание и дисперсия этой величины равны соответственно сумме математических ожиданий и сумме дисперсий двух исходных величин: \(\mu_3 = \mu_1+\mu_2 = 4,25\), 
\(\sigma^2_3 = \sigma^2_1 + \sigma^2_2 = 0,4375.\)

% \(\mu_3 = 3*0,125 + 4*0,5 + 5*0,375 = 4,25\), а дисперсия
% \(\sigma^2_3 = \big( (3-4,25)^2*0,125 + (4-4,25)^2*0,5 + (5-4,25)^2*0,375\big) = 0,4375\).

\paragraph{Задание 4 \textit{(добавлено 2024-10-20)}.}\label{-4.} 

\textit{\textbf{Переделайте скрипт так, чтобы в нем можно было бы анализировать ошибку в оценке вероятности события в зависимости от истинной вероятности и объема выборки.}}

В скрипте истинная вероятность события \verb|true_prob| задается явно в виде одного числа. Далее создаются выборки из n независимых пациентов, каждый из которых излечивается с вероятностью \verb|true_prob|. В качестве выборочной оценки вероятности используется доля излечившихся пациентов. В качестве меры ошибки оценки вероятности я использовал дисперсию выборочных оценок по 1000 выборок. Скрипт позволяет визуализировать распределение выброчных оценок вероятности по отдельности в каждой серии экспериментов (с заданными числом пациентов и истинной вероятностью). Другие визуализации я не делал.

Полный скрипт выложен на \href{https://github.com/ovtarasov/BioStat_2024/blob/main/ProbStat/Tarasov_Prob_HW2-4.Rmd}{github}, ссылка добавлена в гугл-класс.

\textit{\textbf{Какие закономерности вы можете вычислить, экспериментируя со скриптом?}}

Я проанализировал выборки размером от 3 до 1000 пациентов с увеличением приблизительно в 1,5 раза на каждом шаге и значения истинных вероятностей от 0,1 до 0,9 с шагом 0,1. С увеличением объема выборки ошибка монотонно уменьшается, чем дальше --- тем медленнее, зависимость на вид похожа на $1/\sqrt{x}$. Зависимость ошибки от истинной вероятности выпуклая немонотонная и напоминает многочлен второй степени с отрицательным коэффициентом при старшей степени: при истинных вероятностях 0 и 1 ошибка при выбранном методе оценики, очевидно, будет нулевой (в скрипте я это не демонстрирую), а максимум достигается при истинной вероятности 0,5.

\end{document}
